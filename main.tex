\documentclass{sig-alternate-05-2015}
\begin{document}

\setcopyright{acmcopyright}

\title{Improving Collaborative Filtering with Pseudo Side Information}

\numberofauthors{1}
\author{
\alignauthor
Chao Lv \quad
Lili Yao \quad
Yansong Feng \quad
Dongyan Zhao\titlenote{Corresponding author.}\\
\affaddr{Institute of Computer Science and Technology}\\
\affaddr{Peking University, Beijing 100871, China}\\
\email{\{lvchao, yaolili, fengyansong, zhaodongyan\}@pku.edu.cn}
}

\maketitle

\begin{abstract}
Collaborative filtering (CF) has been widely employed within recommender systems in many real-world situations.
Its basic assumption is that items liked by the same user would be similar or users like same items would share similar interest.
But it not always holds because users' interest may change over time.
If user likes two items at the same time period, there is a strong possibility that they are similar.
But the possibility will become small if user likes they at different time point.
In this paper, we propose a method that takes advantage of user ratings timeline to describe the time sensitive relationship between users and items.
To reach this goal, we use a language-based algorithm to learn effective latent embeddings for users and items.
Language model aims to extract the potential association between sentences and words, which is similar with users and items in recommender system.
The learned embedding for users and items can be considered as a kind of side information that describes the time sensitive relationship between users and items.
We then combine them with origin rating information to predict missing ratings in a feature-based collaborative filtering framework.
Experimental results on three MovieLens datasets demonstrate that our approach can achieve the state-of-the-art performance.
\end{abstract}

\category{H.2.8}{Database Management}{Data Mining}
\category{H.3.3}{Information Search and Retrieval}{Information Filtering}
\terms{Algorithms, Experimentation, Performance}
\keywords{Recommender System; Collaborative Filtering; Embedding Model;}

\section{Introduction}
Recommender system has become more and more popular in many real-world situations, in the modern era of information overload.
Lots of websites (e.g. Amazon, Netflix, Alibaba and Hulu) use recommender system to target customers and provide them with useful information.
Recommender system aims to help users find the items, they are more likely to be interested in, from huge amounts of candidates.
A widely used setting of recommendation system is to predict how a user would rate an item (such as a movie) given the past rating history of the users.
Many classical recommendation methods have been proposed in recent years and they can be categorized into three classes:
content-based methods, collaborative filtering (CF) based methods, and hybrid methods.
Content-based methods \cite{pazzani2007content} make use of user profiles or product descriptions for recommendation.
CF-based methods \cite{su2009survey} use the past activities or preferences, such as user ratings on items, without using user or product content information.
Hybrid methods \cite{wang2011collaborative} seek to get the best of both worlds by combining content-based and CF-based methods.
The CF based methods have been developed for many years and keep to be a hot area in both academia and industry due to their impressive performance.
Collaborative filtering focuses on predicting the preference of one user by combining his feedback on a few items and the feedback of all other users.
Among various CF based methods, matrix factorization (MF) models have become popular and achieves the state-of-the-art performance \cite{koren2009matrix}.

\section{Relate Work}
\subsection{Matrix Factoriztion}
\subsection{Nerual Network}

\section{Our Approach}
Local Interest
Global Interest





In this section we present our algorithm for joint modeling of users and the items he rated,
where we learn distributed representations for both the users and the items in a shared and low-dimensional embedding space.
The approach is inspired by \cite{le2014distributed} for learning vector representations of words which take advantage of a word order observed in a sentence.

Then we generate user features and item features via the learned embedding model,
and combine them with the whole origin rating information in the machine learning framework.
Two machine learning methods are tried: regression learning and pairwise ranking.

\subsection{Embedding Model}





\subsection{Machine Learning}
\subsubsection{Regression Learning }
we use \cite{chen2012svdfeature}.

\subsubsection{Pairwise Ranking}

\section{Experiment}
In this section, we conduct several experiments to evaluate the effectiveness of our embedding model.
In these experiments, we also conduct corresponding analysis to investigate:
(1) the influence of the various feedback entity model parameter settings on
retrieval performance;
(2) the effects of feedback tweets number;
(3) the influence of the interpolation coefficient in query expansion and
(4) a comparison of two different entity feedback acquisition methods.



\begin{table*}[htpb]
	\centering
	\caption{Statistics of datasets used in our experiment.}
	\label{tab:topics}
	\begin{tabular}{|l|c|c|c|c|c|c|}
		\hline
		\textbf{Dataset} & \textbf{\#Users} & \textbf{\#Items} & \textbf{\#Ratings} & \textbf{Sparsity} & \textbf{User Features} & \textbf{Item Features} \\
		\hline
		MovieLens-1m  & 6,040   & 3,706  & 1,000,209  & 95.53\% & Gender, Age, and occupation & Genres \\
		MovieLens-10m & 69,878  & 10,677 & 10,000,054 & 98.66\% & - & Genres \\
		MovieLens-20m & 138,493 & 26,744 & 20,000,263 & 99.46\% & - & Genres \\
		\hline
	\end{tabular}
\end{table*}



We employ the root mean squared error (RMSE) and mean absolute error (MAE) as the evaluation metric.
RMSE and MAE are defined as:
$$ RSME = \sqrt{ \frac{1}{N} \sum_{i,j} I_{ij} (R_{ij} - \hat{R}_{ij})^2 } $$
$$ MAE = \frac{1}{N} \sum_{i,j} I_{ij} |R_{ij} - \hat{R}_{ij}| $$
where $N$ is the total number of ratings in the test set,
$R_{ij}$ is the ground-truth rating of user $i$ for item $j$,
$\hat{R_{ij}}$ denotes the corresponding predicted rating,
and $I_{ij}$ is abinary matrix that indicates the ratings in the test set.



\subsection{Movie Recommendation}
For movie recommendation, we conduct experiments on two benchmark
datasets MovieLens-100K and MovieLens-1M 1, which are
commonly used for evaluating collaborative filtering algorithms.
The MovieLens-100K dataset contains 100K ratings of 943 users
and 1682 movies, and the MovieLens-1M dataset consists of about
1 million ratings of 6040 users and 3706 movies. Each rating is
an integer between 1 (worst) and 5 (best). The ratings are highly
sparse. Table 2 summarizes the statistics of datasets. We extract
the features from side information of users and movies to construct
X and Y . To summarize, the user information which consists of
the user‘s age, gender and occupation were encoded into a binary
valued vector of length 28. Similarly, the item feature information
which consists of the 18 category of movie genre were encoded into
a binary valued vector of length 18. Ratings were normalized to be
zero-mean.

\section{Conclusions}
This is the abstract about the paper this is the abstract about the paper.
This is the abstract about the paper this is the abstract about the paper.
This is the abstract about the paper this is the abstract about the paper.
This is the abstract about the paper this is the abstract about the paper.

\section{Acknowledgments}
The work reported in this paper is supported by the National Natural Science Foundation of China Grant 61370116.
We thank anonymous reviewers for their beneficial comments.
We also thank Feifan Fan and Yue Fei for valuable suggestions related to this paper.

\bibliographystyle{abbrv}
\bibliography{sigproc}

\end{document}
