\documentclass{sig-alternate-05-2015}
\begin{document}

\setcopyright{acmcopyright}

\title{Improving Collaborative Filtering with Embedding Model}

\numberofauthors{1}
\author{
\alignauthor
Chao Lv \quad
Lili Yao \quad
Yansong Feng \quad
Dongyan Zhao\titlenote{Corresponding author.}\\
\affaddr{Institute of Computer Science and Technology}\\
\affaddr{Peking University, Beijing 100871, China}\\
\email{\{lvchao, yaolili, fengyansong, zhaodongyan\}@pku.edu.cn}
}

\maketitle

\begin{abstract}
Collaborative filtering (CF) has been widely employed within recommender systems in many real-world situations.
Learning effective latent factors plays the most important role in collaborative filtering.
Traditional CF methods based upon matrix factorization techniques learn the latent factors from the user-item ratings and suffer from the cold start problem as well as the sparsity problem.
Some improved CF methods enrich the priors on the latent factors by incorporating side information as regularization.
However, the learned latent factors may not be very effective due to the sparse nature of the ratings and the side information.
To tackle this problem, we learn effective latent representations via deep learning.
Deep learning models have emerged as very appealing in learning effective representations in many applications.
In particular, we propose a general deep architecture for CF by integrating matrix factorization with deep feature learning.
We provide a natural instantiations of our architecture by combining probabilistic matrix factorization with marginalized denoising stacked auto-encoders.
The combined framework leads to a parsimonious fit over the latent features as indicated by its improved performance in comparison to prior state-of-art models over four large datasets for the tasks of movie/book recommendation and response prediction.
\end{abstract}

\category{H.2.8}{Database Management}{Data Mining}
\category{H.3.3}{Information Search and Retrieval}{Information Filtering}
\terms{Algorithms, Experimentation, Performance}
\keywords{Collaborative Filtering; Embedding}

\section{Introduction}
Recommender system, which recommends items based on users’ interests, has become more and more popular in many real-world situations.
Collaborative filtering (CF) techniques, as the main thrust behind recommender systems, have been developed for many years and keep to be a hot area in both academia and industry.

\section{Relate Work}
\subsection{Matrix Factoriztion}


\subsection{Nerual Network}
Neural Networks have attracted little attention in the CF community.
In a preliminary work, (Salakhutdinov et al., 2007) tackled the Netflix challenge using Restricted Boltzmann Machines but little published work had follow (Truyen et al., 2009).
While Deep Learning has tremendous success in image and speech recognition (LeCun
et al., 2015), sparse data has received less attention and remains a challenging problem for Neural Networks.

Nevertheless, Neural Networks are able to discover nonlinear latent variables with heterogeneous data (LeCun et al., 2015) which makes them a promising tool for CF.
(Sedhain et al., 2015; Strub et al., 2015; Dziugaite et al., 2015) directly train Autoencoders to provide the best predicted ratings.
Those methods report excellent results in the general case.
However, the cold start initialization problem is ignored.
For instance, AutoRec (Sedhain et al., 2015) replaces unpredictable ratings by an arbitrary selected score.
In our case, we apply a training loss designed for sparse rating inputs and we integrate side information to lessen the cold start effect.

Other contributions deal with this cold start problem by using Neural Networks properties for Content-Based Filtering: Neural Networks are first trained to learn a feature representation from the item which is then processed obtain a CF approach such as Probabilistic Matrix Factorization (Mnih et al., 2007) to provide the final rating.
For instance, (Glorot et al., 2011; Wang et al., 2014a) respectively auto-encode bag-of-words from restaurant reviews and movie plots, (Li et al., 2015) auto-encode heterogeneous side information from users and items.
Finally, (den Oord et al., 2013; Wang et al., 2014b) use Convolutional Networks on music samples.
In our case, side information and ratings are used together without any unsupervised
pretreatment.

\section{Method}
This is the abstract about the paper this is the abstract about the paper.

\subsection{SubMethod}
This is the abstract about the paper this is the abstract about the paper.
This is the abstract about the paper this is the abstract about the paper.
This is the abstract about the paper this is the abstract about the paper.
This is the abstract about the paper this is the abstract about the paper.

\subsection{SubMethod}
This is the abstract about the paper this is the abstract about the paper.
This is the abstract about the paper this is the abstract about the paper.
This is the abstract about the paper this is the abstract about the paper.
This is the abstract about the paper this is the abstract about the paper.

\section{Experiment}
In this section, we conduct several experiments to evaluate the effectiveness of our embedding model.
In these experiments, we also conduct corresponding analysis to investigate:
(1) the influence of the various feedback entity model parameter settings on
retrieval performance;
(2) the effects of feedback tweets number;
(3) the influence of the interpolation coefficient in query expansion and
(4) a comparison of two different entity feedback acquisition methods.

\begin{table}[htpb]
	\centering
	\caption{Statistics of datasets used in our experiment.}
	\label{tab:topics}
	\begin{tabular}{|l|c|c|c|c|}
		\hline
		\textbf{Dataset} & \textbf{\#Users} & \textbf{\#Items} & \textbf{\#Ratings} & \textbf{Sparsity} \\
		\hline
		MovieLens-1m  & 6040   & 3706  & 1000209  & 95.53\% \\
		MovieLens-10m & 69878  & 10677 & 10000054 & 98.66\% \\
		MovieLens-20m & 138493 & 26744 & 20000263 & 99.46\% \\
		\hline
	\end{tabular}
\end{table}



\subsection{Movie Recommendation}
For movie recommendation, we conduct experiments on two benchmark
datasets MovieLens-100K and MovieLens-1M 1, which are
commonly used for evaluating collaborative filtering algorithms.
The MovieLens-100K dataset contains 100K ratings of 943 users
and 1682 movies, and the MovieLens-1M dataset consists of about
1 million ratings of 6040 users and 3706 movies. Each rating is
an integer between 1 (worst) and 5 (best). The ratings are highly
sparse. Table 2 summarizes the statistics of datasets. We extract
the features from side information of users and movies to construct
X and Y . To summarize, the user information which consists of
the user‘s age, gender and occupation were encoded into a binary
valued vector of length 28. Similarly, the item feature information
which consists of the 18 category of movie genre were encoded into
a binary valued vector of length 18. Ratings were normalized to be
zero-mean.

\section{Conclusions}
This is the abstract about the paper this is the abstract about the paper.
This is the abstract about the paper this is the abstract about the paper.
This is the abstract about the paper this is the abstract about the paper.
This is the abstract about the paper this is the abstract about the paper.

\section{Acknowledgments}
The work reported in this paper is supported by the National Natural Science Foundation of China Grant 61370116.
We thank anonymous reviewers for their beneficial comments.
We also thank Feifan Fan and Yue Fei for valuable suggestions related to this paper.

\bibliographystyle{abbrv}
\bibliography{sigproc}

\end{document}
